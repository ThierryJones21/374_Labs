\documentclass{article}
\usepackage[a4paper, total={6in, 9in}]{geometry}
\usepackage{graphicx}
\usepackage{xcolor}
\usepackage{listings}

\definecolor{vgreen}{RGB}{104,180,104}
\definecolor{vblue}{RGB}{49,49,255}
\lstset{
    language=Verilog,
    % wrap text
    breaklines=true, 
    % line numbers
    numbers=left,
    numberstyle=\tiny\color{black},
    numbersep=10pt,
    % other styling
    basicstyle=\small\ttfamily,
    keywordstyle=\color{vblue},
    identifierstyle=\color{black},
    commentstyle=\color{vgreen},
    tabsize=2
}
\graphicspath{ {Images/} }

\title{
    \begin{large}
        ELEC374 - Lab 3
    \end{large}
}
\author{Naod Dereje - 20103501, Thierry Jones - 20108349}

\begin{document}
\maketitle
\cleardoublepage
\tableofcontents
\cleardoublepage

\section{Components}
    The purpose of this lab was to design, simulate, implement and verify a simple RISC computer(Mini SRC). 
    Verilog was chosen over VHDL as it is better for more complex simulations. The verilog code for each of these components can be found in the Appendices.
    
    \subsection{Datapath}
    Select and Encode logic was used for load and store instruction as well as add, and, or instructions. The opcode that is read into Select and Encode is used to create the outputs for Ra, Rb , and Rc, in order to generate the GRa, GRb, GRc, which is the encoded to R0in-R15in and R0out-R15out registers. Their is also support for a sign-extended C value in the lower 19 bits of the instrcution address.

    \subsection{Control Unit}
    The control unit was used to compute all the necessary instructions using signal in order for there to be only one testbench needed.
    In order to define the control unit all datapath signals were passed from previous labs and new ones in the ALU. Additionally, the RAM's memory is 
    accessed through the corresponding hex file, Which contains the opcodes for all instructions.
    A detailed breakdown of the memory subsystem can be found at Appendix \ref{control_unit.v}.

    \subsection{ALU}
    The conff logic is created to ensure that conidtional logic such as the branch insturction are able to be executed. The instructions assocaited with the conditional logic 
    are stored in I-formatting, with the second register holding the branching condition. The components for the con ff logic can be found at Appendix \ref{con_ff.v}.

    
\section{Circuitry Demonstration}
    To demonstrate the success of the RAM and memory interface logic, multiple testbenches were created to simulate each operation.  The testbenches all have their own control sequences unique to every operation and can be found in the appendices. The following subsections of the cicuitry demonstration will detail the changes made in the test benches due to the control sequence changing as well as the waveforms generated by each simulation. Each instruction used to demonstrate the circuitry requires a specific OP Code. The following table shows the OP Code values stored in RAM initially, their address in RAM and also the instruction itself. For a full version of the initial RAM state, please see Appendix \ref{ram_init}. Furthermore, in an effort to save paper, only the relevant signals were included for each instruction.
    \begin{figure}[h!] \label{relevant_ram_init}
        \begin{center}
            \begin{tabular}{|c|c|c|c|}
                \hline
                Address & Instruction & Op code (binary) & Op code (hex) \\
                \hline
                0 & ldi R3, \$87 & 00001 0011 0000 000 0000 0000 1000 1000 & 09800087 \\
                \hline
                1 & ldi R3, 1(R3) & 00001 0011 0011 000 0000 0000 0000 0001 & 09980001 \\
                \hline
                2 & ld R2, \$75 & 00000 0010 0000 000 0000 0000 0111 0101 & 01000075 \\
                \hline
                3 & ldi R2, -2(R2) & 00001 0010 0010 111 1111 1111 1111 1110 & 0917FFFE \\
                \hline
                4 & ld R1, 4(R2) & 00000 0001 0010 000 0000 0000 0000 0100 & 00900004 \\
                \hline
                5 & ldi R0, 1 & 00001 0000 0000 000 0000 0000 0000 0001 & 08000001 \\
                \hline
                6 & ldi R3, \$73 & 00001 0011 0000 000 0000 0000 0111 0011 & 09800073 \\
                \hline
                7 & brmi R3, 3 & 10010 0011 0000 000 0000 0000 0000 0011 & 91800003 \\
                \hline
                8 & ldi R3, 5(R3) & 00001 0011 0011 000 0000 0000 0000 0101 & 09980005 \\
                \hline
                9 & ld R7, -3(R3) & 00000 0111 0011 111 1111 1111 1111 1101 & 039FFFFD \\
                \hline
                10 & nop & 11001 0000 0000 000 0000 0000 0000 0000 & C8000000 \\
                \hline
                11 & brpl R7, 2 & 10010 0111 0010 000 0000 0000 0000 0010 & 93900002 \\
                \hline
                12 & ldi R4, 6(R1) & 00001 0100 0001 000 0000 0000 0000 0110 & 0A080006 \\
                \hline
                13 & ldi R3, 2(R4) & 00001 0011 0100 000 0000 0000 0000 0010 & 09A00002 \\
                \hline
                14 & add R3, R2, R3 & 00011 0011 0010 0011 000 0000 0000 0000 & 19918000 \\
                \hline
                15 & addi R7, R7, 3 & 01011 0111 0111 000 0000 0000 0000 0011 & 5BB80003 \\
                \hline
                16 & neg R7, R7 & 10000 0111 0111 000 0000 0000 0000 0000 & 83B80000 \\
                \hline
                17 & not R7, R7 & 10001 0111 0111 000 0000 0000 0000 0000 & 8BB80000 \\
                \hline
                18 & andi R7, R7, \$0F & 01100 0111 0111 000 0000 0000 0000 1111 & 63B8000F \\
                \hline
                19 & ori R7, R1, 3 & 01101 0111 0001 000 0000 0000 0000 0011 & 6B880003 \\
                \hline
                20 & shr R2, R3, R0 & 00101 0010 0011 0000 000 0000 0000 0000 & 29180000 \\
                \hline
                21 & st \$58, R2 & 00010 0010 0000 000 0000 0000 0101 1000 & 11000058 \\
                \hline
                22 & ror R1, R1, R0 & 00111 0001 0001 0000 000 0000 0000 0000 & 38880000 \\
                \hline
                23 & rol R2, R2, R0 & 01000 0010 0010 0000 000 0000 0000 0000 & 41100000 \\
                \hline 
                24 & or R2, R3, R0 & 01010 0010 0011 0000 000 0000 0000 0000 & 51180000 \\
                \hline
                25 & and R1, R2, R1 & 01001 0001 0010 0001 000 0000 0000 0000 & 48908000 \\
                \hline
                26 & st \$67(R1), R2 & 00010 0010 0001 000 0000 0000 0110 0111 & 11080067 \\
                \hline
                27 & sub R3, R2, R3 & 00100 0011 0010 0011 000 0000 0000 0000 & 21918000 \\
                \hline
                28 & shl R1, R2, R0 & 00110 0001 0010 0000 000 0000 0000 0000 & 30900000 \\
                \hline
                29 & ldi R4, 5 & 00001 0100 0000 000 0000 0000 0000 0101 & 0A000005 \\
                \hline
                30 & ldi R5, \$1D & 00001 0101 0000 000 0000 0000 0001 1101 & 0A80001D \\
                \hline
                31 & mul R5, R4  & 01110 0101 0100 000 0000 0000 0000 0000 & 72A00000 \\
                \hline
                32 & mfhi R7 & 10111 0111 0000 000 0000 0000 0000 0000 & BB800000 \\
                \hline
                33 & mflo R6 & 11000 0110 0000 000 0000 0000 0000 0000 & C3000000 \\
                \hline
                34 & div R5, R4 & 01111 0101 0100 000 0000 0000 0000 0000 & 7AA00000 \\
                \hline
                35 & ldi R10, 0(R4) & 00001 1010 0100 000 0000 0000 0000 0000 & 0D200000 \\
                \hline
                36 & ldi R11, 2(R5) & 00001 1011 0101 000 0000 0000 0000 0010 & 0DA80002 \\
                \hline
                37 & ldi R12, 0(R6) & 00001 1100 0110 000 0000 0000 0000 0000 & 0E300000 \\
                \hline
                38 & ldi R13, 0(R7) & 00001 1101 0111 000 0000 0000 0000 0000 & 0EB80000 \\
                \hline
                39 & jal R12 & 10100 1100 0000 000 0000 0000 0000 0000 &A6700000 \\
                \hline
                40 & halt & 11010 0000 0000 000 0000 0000 0000 0000 &D0000000 \\
                \hline
                41 & ORG \$91: & – & –\\
                \hline
                145 &add R9, R10, R12 & 00011 1001 1010 1100 000 0000 0000 0000 & 1CD60000\\
                \hline
                146 &sub R8, R11, R13 & 00100 1000 1011 1101 000 0000 0000 0000 & 245E8000\\
                \hline
                147 & sub R9, R9, R8 & 00100 1001 1001 1000 000 0000 0000 0000 & 24CC0000\\
                \hline
                148 & jr R14 & 10011 1110 1000 0000 000 0000 0000 0000 & 9F800000\\
                \hline
            \end{tabular}
            \caption{Table of relevant values in RAM initially}
        \end{center}
    \end{figure}
    \\
    \small{*Note that these instructions were tested before the rest, thus, the .hex file hadn't be completed. Thus, Appendix \ref{ram_st_case_1} and Appendix \ref{ram_st_case_2} have empty values in RAM where the OP Codes are for later tests.}

    \subsection{Load Instructions}
        Two load instructions were tested to ensure that the values are able to load from the RAM onto its appropriate register. The control sequences for \emph{ld R1,\$85}, \emph{ld R0, \$35(R1)}, \emph{ldi R1,\$85}, \emph{ldi R0, \$35(R1)} can be found in Appendices \ref{ld_case_1_sequence}, \ref{ld_case_2_sequence}, \ref{ldi_case_1_sequence}, and \ref{ldi_case_2_sequence} respectively. The value $85_{16}$ and $35_{16}$ was preloaded into the memory initialization file in slots $85_{16}$ and $35_{16}$ for these instructions respectively. The specific opcodes for these instructions can be found in Figure \ref{relevant_ram_init}, and the relevant address was loaded into the PC register. The value of $0_{16}$ was preloaded into register 1.
       
       

\cleardoublepage

\appendix
\section{General Components}
    \subsection{Control Unit} \label{control_unit.v}
        \lstinputlisting{control_unit.v}
    \subsection{Datapath} \label{datapath.v}
        \lstinputlisting{datapath.v}
    \subsection{Bus} \label{Bus.v}
        \lstinputlisting{Bus.v} 
    \subsection{Register Zero} \label{register_zero.v}
        \lstinputlisting{register_zero.v} 
    \subsection{Arithmetic Logic Unit} \label{ALU.v}
        \lstinputlisting{ALU.v}
\section{Select and Encode}
    \lstinputlisting{select_and_encode.v} \label{select_and_encode.v}
\section{Memory Subsystem}
    \subsection{RAM} \label{ram.v}
        \lstinputlisting{ram.v} 
    \subsection{Memory Address Register} \label{MAR.v}
        \lstinputlisting{MAR.v}
    \subsection{Memory Data Register} \label{MDR.v}
        \lstinputlisting{MDR.v} 
\section{CON FF} \label{con_ff.v}
    \lstinputlisting{con_ff.v} 
\section{Input/Output Ports}
    \subsection{Input Port} \label{inputPort.v}
        \lstinputlisting{inputPort.v}
    \subsection{Output Port} \label{outputPort.v}
        \lstinputlisting{outputPort.v}
    
\section{Control Sequences}
    \subsection{st \$90, R1} \label{st_case_1_sequence}
        \lstinputlisting{InstructionTestBenches/st_case_1_TB.v}
    \subsection{st \$90(R1), R1} \label{st_case_2_sequence}
        \lstinputlisting{InstructionTestBenches/st_case_2_TB.v}
    \subsection{ld R1, \$85), R1} \label{ld_case_1_sequence}
        \lstinputlisting{InstructionTestBenches/ld_case_1_TB.v}
    \subsection{ld R0, \$35(R1)} \label{ld_case_2_sequence}
        \lstinputlisting{InstructionTestBenches/ld_case_2_TB.v}
    \subsection{ldi R1, \$85} \label{ldi_case_1_sequence}
        \lstinputlisting{InstructionTestBenches/ldi_case_3_TB.v}
    \subsection{ldi R0, \$35(R1)} \label{ldi_case_2_sequence}
        \lstinputlisting{InstructionTestBenches/ldi_case_4_TB.v}
    \subsection{brzr R2, 35} \label{brzr_sequence}
        \lstinputlisting{InstructionTestBenches/brzr_TB.v}
    \subsection{brnz R2, 35} \label{brnz_sequence}
        \lstinputlisting{InstructionTestBenches/brnz_TB.v}
    \subsection{brpl R2, 35} \label{brpl_sequence}
        \lstinputlisting{InstructionTestBenches/brpl_TB.v}
    \subsection{brmi R2, 35} \label{brmi_sequence}
        \lstinputlisting{InstructionTestBenches/brmi_TB.v}
    \subsection{jal R1} \label{jal_sequence}
        \lstinputlisting{InstructionTestBenches/jal_TB.v}
    \subsection{jr R1} \label{jr_sequence}
        \lstinputlisting{InstructionTestBenches/jr_TB.v}
    \subsection{in R1} \label{in_sequence}
        \lstinputlisting{InstructionTestBenches/in_TB.v}
    \subsection{out R1} \label{out_sequence}
        \lstinputlisting{InstructionTestBenches/out_TB.v}
    \subsection{addi R2, R1, -5} \label{addi_sequence}
        \lstinputlisting{InstructionTestBenches/addi_TB.v}
    \subsection{andi R2, R1 \$26} \label{andi_sequence}
        \lstinputlisting{InstructionTestBenches/andi_TB.v}
    \subsection{ori R2, R1 \$26} \label{ori_sequence}
        \lstinputlisting{InstructionTestBenches/ori_TB.v}
    \subsection{mfhi R2} \label{mfhi_sequence}
        \lstinputlisting{InstructionTestBenches/mfhi_TB.v}
    \subsection{mflo R2} \label{mflo_sequence}
        \lstinputlisting{InstructionTestBenches/mflo_TB.v}

    


\end{document}
