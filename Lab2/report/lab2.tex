\documentclass{article}
\usepackage[a4paper, total={6in, 9in}]{geometry}
\usepackage{graphicx}
\usepackage{xcolor}
\usepackage{listings}
\definecolor{vgreen}{RGB}{104,180,104}
\definecolor{vblue}{RGB}{49,49,255}
\lstset{
    language=Verilog,
    % wrap text
    breaklines=true, 
    % line numbers
    numbers=left,
    numberstyle=\tiny\color{black},
    numbersep=10pt,
    % other styling
    basicstyle=\small\ttfamily,
    keywordstyle=\color{vblue},
    identifierstyle=\color{black},
    commentstyle=\color{vgreen},
    tabsize=2
}
\graphicspath{ {images/} }

\title{
    \begin{large}
        ELEC374 - Lab 2
    \end{large}
}
\author{Naod Dereje - 20103501, Thierry Jones - 20108349, Jamie Won - 20113217}

\begin{document}
\maketitle
\cleardoublepage
\tableofcontents
\cleardoublepage


\section{Components}
    The purpose of this lab was to design, simulate, implement and verify a simple RISC computer(Mini SRC), including the circuits associated with the logic behind: 
    Select and Encode, Memory Subsystem, CON FF and Input/Output ports. Verilog was chosen over VHDL as it is better for more complex simulations. The verilog code 
    for each of these components can be found in the Appendices.

    \subsection{Select and Encode}


    \subsection{Memory Subsystem}
    

    \subsection{CON FF}
    

    \subsection{Input/Output Ports}
    
\section{Circuitry Demonstration}
    To demonstrate the success of the RAM and memory interface logic, multiple testbenches were created to simulate each operation. 
    The testbenches all have their own control sequences unique to every operation and can be found in the appendices. The following subsections 
    of the cicuitry demonstration will detail the changes made in the test benches due to the control sequence changing as well as the waveforms generated by each simulation.

\appendix
\section{Select and Encode}
\section{Memory Subsystem}
\section{CON FF}
\section{Input/Output Ports}
\section{Control Sequences}

\end{document}
