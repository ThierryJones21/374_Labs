\documentclass{article}
\usepackage[a4paper, total={6in, 9in}]{geometry}
\usepackage{graphicx}
\usepackage{multicol}
\usepackage{listings}
\lstset{
    language=Verilog,
    breaklines=true,
    tabsize=2
}
\graphicspath{ {images/} }

\title{
    \begin{large}
        ELEC374 - Lab 1
    \end{large}
}
\author{Naod Dereje - 20103501, Thierri Jones - 20108349, Jamie Won - 20113217}

\begin{document}
\maketitle
\cleardoublepage
\tableofcontents
\cleardoublepage

\section{Components}
    The purpose of this lab was to design and simulate a part of the mini SRC datapath. A bus, MDR, ALU, datapath and 16 registers were needed to do so. Verilog was chosen over VHDL as it is better for more complex simulations. The verilog code for each of these components can be found in the Appendices.

    \subsection{Bus}
    The bus was implemented using a 32-to-5 bit encoder. The verilog code for this component can be found in Appendix \ref{Bus}

    \subsection{Memory Data Register (MDR)}
    The memory data register was implemented with a 32 bit multiplexer and flip flop. The code for the flip flop can be found below.
        \lstinputlisting{Modules/dff_32bit.v}
    The code for the multiplexer can be found below.
        \lstinputlisting{Modules/mux21_32bit.v}
    Finally, the code for the MDR itself can be found in Appendix \ref{MDR}

    \subsection{Datapath}
    The datapath module connects all the other modules. There are some internal connections that were made, as well as 16 registers that were created. The code for a single register can be found below.
        \lstinputlisting{Modules/Register.v}
    The verilog code for the datapath can be found in Appendix \ref{Datapath}.

    \subsection{Arithmetic Logic Unit (ALU)}
    The ALU is able to perform addition, subtraction, multiplication, division, not, negation, and, shifting left and right, and rotating left and right. The verilog code for these operations as well as the ALU itself can be found in Appendix \ref{ALU}.
\section{Circuitry Demonstration}
    \subsection{and R5, R2, R4}
    \subsection{or R5, R2, R4}
    \subsection{add R5, R2, R4}
    \subsection{sub R5, R2, R4}
    \subsection{mul R5, R2, R4}
    \subsection{div R5, R2, R4}
    \subsection{shr R5, R2, R4}
    \subsection{shl R5, R2, R4}
    \subsection{ror R5, R2, R4}
    \subsection{rol R5, R2, R4}
    \subsection{neg R5, R2, R4}
    \subsection{not R5, R2, R4}

\appendix
% \section{Testbench}
% \lstinputlisting{Lab1.v}
\section{Bus} \label{Bus}
    \lstinputlisting{Bus.v}
\section{MDR}\label{MDR}
    \lstinputlisting{MDR.v}
\section{Datapath} \label{Datapath}
    \lstinputlisting{datapath.v}
\section{ALU} \label{ALU}
    \lstinputlisting{ALU.v}
    \subsection{And}
    \subsection{Or}
    \subsection{Add}
    \subsection{Subtract}
    \subsection{Multiplication}
    \subsection{Division}
    \subsection{Negation}
    \subsection{Not}
    \subsection{Shift Left}
    \subsection{Shift Right}
    \subsection{Rotate Left}
    \subsection{Rotate Right}
\end{document}