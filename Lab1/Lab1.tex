\documentclass{article}
\usepackage[a4paper, total={6in, 9in}]{geometry}
\usepackage{graphicx}
\usepackage{xcolor}
\usepackage{listings}
\definecolor{vgreen}{RGB}{104,180,104}
\definecolor{vblue}{RGB}{49,49,255}
\lstset{
    language=Verilog,
    % wrap text
    breaklines=true, 
    % line numbers
    numbers=left,
    numberstyle=\tiny\color{black},
    numbersep=10pt,
    % other styling
    basicstyle=\small\ttfamily,
    keywordstyle=\color{vblue},
    identifierstyle=\color{black},
    commentstyle=\color{vgreen},
    tabsize=2
}
\graphicspath{ {images/} }

\title{
    \begin{large}
        ELEC374 - Lab 1
    \end{large}
}
\author{Naod Dereje - 20103501, Thierry Jones - 20108349, Jamie Won - 20113217}

\begin{document}
\maketitle
\cleardoublepage
\tableofcontents
\cleardoublepage

\section{Components}
    The purpose of this lab was to design and simulate a part of the mini SRC datapath. A bus, MDR, ALU, datapath and 16 registers were needed to do so. Verilog was chosen over VHDL as it is better for more complex simulations. The verilog code for each of these components can be found in the Appendices.

    \subsection{Bus}
    The bus was implemented using a 32-to-5 bit encoder. The verilog code for this component can be found in Appendix \ref{Bus}

    \subsection{Memory Data Register (MDR)}
    The memory data register was implemented with a 32 bit multiplexer and flip flop. The code for the flip flop can be found below.
        \lstinputlisting{Modules/dff_32bit.v}
    The code for the multiplexer can be found below.
        \lstinputlisting{Modules/mux21_32bit.v}
    Finally, the code for the MDR itself can be found in Appendix \ref{MDR}

    \subsection{Datapath}
    The datapath module connects all the other modules. There are some internal connections that were made, as well as 16 registers that were created. The code for a single register can be found below.
        \lstinputlisting{Modules/Register.v}
    The verilog code for the datapath can be found in Appendix \ref{Datapath}.

    \subsection{Arithmetic Logic Unit (ALU)}
    The ALU is able to perform addition, subtraction, multiplication, division, not, negation, and, shifting left and right, and rotating left and right. The verilog code for these operations as well as the ALU itself can be found in Appendix \ref{ALU}.

\section{Circuitry Demonstration}

    To demonstrate the success of the ALU, multiple testbenches were created to simulate each operation. Barring the control sequence, the testbenches are all identical to the one found in Appendix 
    % \ref{Testbench} 
    \textbf{After NAOD's PUSH, fix characters and  UNCOMMENT THE LINE ABOVE}
    The following subsections of the report detail the changes made in the test bench due to the control sequence changing as well as the waveforms generated by each simulation.

    \subsection{and R5, R2, R4}
    This instruction demonstrates the ALU's and circuitry. The control sequence for this instruction was provided in the lab manual. The control sequence for this instruction can be found below in Appendix \ref{AND}.
    \subsection{or R5, R2, R4}
    \subsection{add R5, R2, R4}
    \subsection{sub R5, R2, R4}
    \subsection{mul R5, R2, R4}
    \subsection{div R5, R2, R4}
    \subsection{shr R5, R2, R4}
    \subsection{shl R5, R2, R4}
    \subsection{ror R5, R2, R4}
    \subsection{rol R5, R2, R4}
    \subsection{neg R5, R2, R4}
    \subsection{not R5, R2, R4}

\appendix
% \section{Testbench}
% \lstinputlisting{Lab1.v}
\section{Bus} \label{Bus}
    \lstinputlisting{Bus.v}
\section{MDR}\label{MDR}
    \lstinputlisting{MDR.v}
\section{Datapath} \label{Datapath}
    \lstinputlisting{datapath.v}
\section{ALU} \label{ALU}
    \subsection{ALU}
        \lstinputlisting{ALU.v}
    \subsection{And}
        \begin{lstlisting}
            0   :   C = A & B;
        \end{lstlisting}
    \subsection{Or}
        \begin{lstlisting}
			1   :   C = A | B;
        \end{lstlisting}
    \subsection{Add}
        \begin{lstlisting}
            2   :   C = A + B;
        \end{lstlisting}
    \subsection{Subtract}
        \begin{lstlisting}
			3   :   C = A - B;
        \end{lstlisting}
    \subsection{Multiplication}
        \lstinputlisting{ALU.v}[firstline=3,lastline=58]
    \subsection{Division}
        \begin{lstlisting}
            11  :   C = A/B;  //TO-DO: Make cool division algo
        \end{lstlisting}
    \subsection{Negation}
        \begin{lstlisting}
            4   :   C = -B; //negation function
        \end{lstlisting}
    \subsection{Not}
        \begin{lstlisting}
            5   :   C = !B; // logical not 
        \end{lstlisting}
    \subsection{Shift Left}
        \begin{lstlisting}
            6   :   C = A <<< B; // left arithtmatic shift - A = how many shifts, B = the number you want to shift 
        \end{lstlisting}
    \subsection{Shift Right}
        \begin{lstlisting}
            7   :   C = A >>> B; // right arithmetic shift - A = how many shifts, B = the number you want to shift 
        \end{lstlisting}
    \subsection{Rotate Left}
        \lstinputlisting{ALU.v}[firstline=68,lastline=71]
    \subsection{Rotate Right}
        \lstinputlisting{ALU.v}[firstline=64,lastline=67]
\section{Control Sequences}
    \subsection{and R5, R2, R4} \label{AND}
        \lstinputlisting{TestBenches/tb_AND.v}
    \subsection{or R5, R2, R4} \label{OR}
        \lstinputlisting{TestBenches/tb_OR.v}
    \subsection{add R5, R2, R4} \label{ADD}
        \lstinputlisting{TestBenches/tb_ADD.v}
    \subsection{sub R5, R2, R4} \label{SUB}
        \lstinputlisting{TestBenches/tb_SUB.v}
    \subsection{mul R5, R2, R4} \label{MUL}
        \lstinputlisting{TestBenches/tb_MUL.v}
    \subsection{div R5, R2, R4} \label{DIV}
        \lstinputlisting{TestBenches/tb_DIV.v}
    \subsection{shr R5, R2, R4} \label{SHR}
        \lstinputlisting{TestBenches/tb_SHR.v}
    \subsection{shl R5, R2, R4} \label{SHL}
        \lstinputlisting{TestBenches/tb_SHL.v}
    \subsection{ror R5, R2, R4} \label{ROR}
        \lstinputlisting{TestBenches/tb_ROR.v}
    \subsection{rol R5, R2, R4} \label{ROL}
        \lstinputlisting{TestBenches/tb_ROL.v}
    \subsection{neg R5, R2, R4} \label{NEG}
        \lstinputlisting{TestBenches/tb_NEG.v}
    \subsection{not R5, R2, R4} \label{NOT}
        \lstinputlisting{TestBenches/tb_NOT.v}
\end{document}
